\documentclass{article}
\usepackage[utf8]{inputenc}

\setlength{\arrayrulewidth}{1mm}
\setlength{\tabcolsep}{18pt}
\renewcommand{\arraystretch}{1.5}

\begin{document}
\begin{tabular}{ |p{3cm}|p{3cm}|p{3cm}|p{3cm}|p{3cm}|p{3cm}|p{3cm}|  }
\hline
\multicolumn{7}{|c|}{Nono Luigi - Live Electronics Overview} \\
\hline
Data& Titolo& Ensemble& Ensemble& Live Electronics &Live Electronics  &Live Electronics \\
\hline
1979 & Con Luigi Dallapiccola& Percussioni |  & Ring Modulators &Oscillators & - &- \\
1981 & Das atmende Klarsein &Flauto & Coro & Halafon &Delay &Harmonizer \\
1981 & Io, frammento dal Prometeo&-&-&-&-&- \\
1982 & Quando stanno morendo. Diario polacco n. 2&-&-&-&-&- \\
1983 &Guai ai gelidi mostri &-&-&-&-&- \\
1983 &Omaggio a György Kurtág &-&-&-&-&- \\
1984 &Prometeo. Tragedia dell’ascolto &-&-&-&-&- \\
1985 &A Pierre. Dell’azzurro silenzio, inquietum &-&-&-&-&- \\
1986 &Risonanze erranti. Liederzyklus a Massimo Cacciari &-&-&-&-&-\\
1987 &1° Caminantes…..Ayacucho&-&-&-&-&- \\
1987 &Découvrir la subversion. Hommage à Edmond Jabès &-&-&-&-&- \\
1987 &Post-prae-ludium n. 1 per Donau &-&-&-&-&- \\
\hline
\end{tabular}
\end{document}