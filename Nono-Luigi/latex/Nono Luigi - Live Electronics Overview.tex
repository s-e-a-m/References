\documentclass{article}
\usepackage[a3paper,margin=1in,landscape]{geometry}
\usepackage[utf8]{inputenc}
\usepackage[english]{babel} 
\usepackage{wasysym}

\setlength{\arrayrulewidth}{1mm}
\setlength{\tabcolsep}{18pt}
\renewcommand{\arraystretch}{1.5}

\begin{document}
\begin{tabular}{ |c|c|c|c|c|c|c|}
\hline
\multicolumn{7}{|c|}{Nono Luigi - Live Electronics Overview} \\
\hline
Data& Titolo& Ensemble& Ensemble&  Ensemble& Live Electronics &Live Electronics\\
\hline
1979 & Con Luigi Dallapiccola& Percussioni  & Ring Modulators &Oscillators &-&  \\
1981 & Das atmende Klarsein &Flauto & Coro & Halafon &Delay &Harmonizer \\
1981 & Io, frammento dal Prometeo&3 Soprani&Piccolo coro a 12 voci&Flauto basso, Clarinetto contrabbasso in Si$\flat$&-&-\\
1982 & Quando stanno morendo. Diario polacco n. 2&2 Soprani, Mezzosoprano, Contralto&Flauto basso,Violoncello&-&-&- \\
1983 &Guai ai gelidi mostri &-&-&-&-&- \\
1983 &Omaggio a György Kurtág &-&-&-&-&- \\
1984 &Prometeo. Tragedia dell’ascolto &-&-&-&-&- \\
1985 &A Pierre. Dell’azzurro silenzio, inquietum &-&-&-&-&- \\
1986 &Risonanze erranti. Liederzyklus a Massimo Cacciari &-&-&-&-&-\\
1987 &1º Caminantes…..Ayacucho&-&-&-&-&- \\
1987 &Découvrir la subversion. Hommage à Edmond Jabès &-&-&-&-&- \\
1987 &Post-prae-ludium n. 1 per Donau &-&-&-&-&- \\
\hline
\end{tabular}
\end{document}